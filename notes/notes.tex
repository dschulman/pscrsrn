\documentclass{article}

\title{Physiological Signal Classification with Recursive Sequence Reduction}
\author{Daniel Schulman}

\begin{document}
\maketitle

\section{Introduction}

Classification of physiological signals, such as electromyography (EMG), electroencephalography (EEG), and electrocardiography (ECG), is a common task in applications of artificial intelligence and machine learning to healthcare.
In the last few years, deep learning models have become state-of-the-art in many physiological signal classification tasks, in several cases achieving performance superior to human experts.

Physiological signals differ from many of the domains that deep learning models were originally developed on, such as computer vision.
They typically have fairly high temporal resolution, so that even brief recordings of signals have many more samples than other types of sequential data (e.g. natural language).
Signal length often varies, and models that are designed around data with fixed dimensionality (e.g. CNNs) can be awkward.

In this paper, we introduce a novel approach --- a \emph{recursive sequence reduction network} (RSRNs) --- that provides a simple baseline approach that is well-suited to physiological signal classification tasks.
RSRNs handle long sequences and varying-length sequence, and can deliver competitive performance with relatively small models, which may make usage feasible on edge devices with limited memory or computational power.

We first introduce RSRNs, then present experiments demonstrating performance on a set of benchmark tasks.
We compare and contrast RSRNs to other models, and we discuss applications where RSRNs may be suitable.

\end{document}